\chapter{Mid Project Report}
\centerline{\rule{149mm}{.02in}}
\vspace{2cm}

This project has got off to a good start. From the formulation of the ideas and project plan, to background research, most of the 'preparation' phase of my work, as detailed in the following draft introduction chapter, has been finished. Currently, I am moving on to the practical work required for the 'Implementation phase' of the project. The work I have done so far will be outlined in this section, and provided in full in other sections of this report.\\

\subsection{Part 1: Aims \& Objectives}

The first part of the work I completed was concerned with what the focus of the project would actually be. The title of this project, An Evaluation of OpenStack, left a lot of scope for choice for my approach. These aims \& objectives were decided on in the first week by myself and Karim, based on what we would like to see as an end result for the project, and what we were trying to achieve. The full aims, objectives, minimum requirements, and deliverables have been outlined in the 'introduction' section of this report, along with possible extensions. The basic idea we came up with was to design and execute a number of experiments to assess OpenStack functionality. Part of this was also to justify the project \& outline what it would actually achieve, and this is also covered in the introduction chapter. 

\subsection{Part 2: Project Planning} 
Once the overall aim of the project was decided, the next step was to actually plan the project, from two different perspectives; one from the organisation and scheduling perspective, and one concerning the actual project approach and methodologies I would be using. 

\subsubsection{Project Methodology}

This part of the decision making and planning phase was important to finish first as the actual timetabling of work was dependent on it. This included getting some idea of what work needed to be done, and how it would be approached, including the approach to developing experiments to execute throughout the project. The actual project approach, covered in the introduction chapter, was split into 4 main stages, the first of which includes project planning. Mind maps were also developed, here seen in appendix B, to get an overview of what work was to be done. 

\subsubsection{Project Scheduling}

Once the approach and work to do was decided and written up, I felt it necessary to begin to plan and schedule when certain work would be completed. The first step here was to create a GANTT chart detailing the high level tasks which would need completing, to give some target for each week's achievements. This can be found in appendix B. Similarly, a chart of each project deadline was made to compare this with. So far, this GANTT chart has been strictly adhered to, until very recently, i.e. week starting 03/03/2014, as holdups to the availability of OpenStack on the Cloud TestBed have prevented me from doing any practical work thus far. This could have a knock-on effect for the rest of the project, and so it is likely that a new GANTT chart may be needed, or the original may be deviated from. \\
The next step in this process was to take deadlines and tasks and form a number of milestones to effectively judge the progress of the project; these are detailed in the introduction chapter. 

\subsection{Background Research} 

This has been the main bulk of the work done to date. The literature review and write up, detailed in the Background Research chapter, gives an introduction to the Cloud Computing domain, focusing on the relevant areas to OpenStack, such as Virtual Infrastructure Management \& Infrastructure as a Service clouds. The research focuses on the current OpenStack offering and its competitors, as well as on the desired aspects \& characteristics of a cloud, as well as the challenges they face. The idea behind this is to target experiments at OpenStack in a way which exhibits the desirable characteristics of a cloud, and deals effectively with its challenges, or not as the case may be. The chapter culminates in my overall plan for evaluating OpenStack. 

\subsection{Implementation}

This is the stage I am currently at, and I am currently waiting for access to OpenStack on the testbed. So far however, I have designed one basic experiment which will log in to OpenStack using the identity service, and use the Nove (compute) part of OpenStack to start up a virtual machine. The idea is to perform this experiment with the Command Line Interface, REST APIs, and Web UI of OpenStack so as to compare their effectiveness for basic usage, and to experiment or 'warm up' with OpenStack and its many use cases. I have also began to set up my development environment using Eclipse \& Jersey, which will be documented in the warm-up section of implementation.

\subsection{Next Steps}

Firstly, I will complete my warm-up exercise and have a written comparison of OpenStack's different interfaces. The next step will be to design and implement a number of experiments, then test \& execute them, collect results, and write up the procedure of each experiment with its results. \\
Once this is finished, it will be time to form some conclusions about my results, and come up with some form of evaluation and point of comparison with competitors for OpenStack. \\
Finally, I will evaluate my approach and results, reflecting on the strengths and weaknesses of my approach, and giving some recommendation as to the usefulness of what I have produced, and what it has contributed to its field. 
