\begin{center}
    {\LARGE\bf Summary}
\end{center}

This project involves researching into the various features and characteristics of the OpenStack technology, via a number of different approaches, including research analysis \& both manual and automated experimentation, in order to provide an Evaluation of the technology for the upcoming EU project ASCETiC\cite{ascetic}. \\ \\ This 'Evaluation' of OpenStack will hope to achieve a greater insight to its workings, providing a guide by which to effectively utilise its features whilst avoiding its potential pitfalls and difficulties. It will also provide validation for a lot of key functionality, as a way of determining whether use of OpenStack in a project is justified, along with a review of the various use cases of OpenStack, in order to assess it's viability in the current market. In order to achieve these aims, a simple experiment based model will be used, with logs and conclusions of these experiments forming the bulk of the 'evaluation'. A high level research approach will be used in conjunction with this to give a good overview of the technology. \\ \\ 
The deliverables for this project include a report detailing my approach to evaluating OpenStack, as well as a number of designed experiments with results and logs attached, including my own reflection and conclusions on what these results represent. As a secondary objective, this project aims to provide re-usable software tools and components, such as libraries and  scripts, for the effective use of OpenStack in future. 



