\chapter{Personal Reflection}
\centerline{\rule{149mm}{.02in}}
\vspace{2cm}

I initially chose this project due to a vested interest in the Cloud Computing Domain, from previous work and my planned career path. I thought that, as I had interest in the area, I would be happy to work long hours to achieve a greater level of knowledge and contribution to the field. I was at first concerned that I had made a mistake in choosing in this way, as many consider the Final Year Project a way of exploring and learning something which you may not have another chance to look at, something specific to research or academic environments. This fear was quickly quelled as I realised the enormity of the task ahead; I had definitely not picked the easy option. The work required of me throughout the project stretched me to trying new things, learning new techniques, and ultimately adapting to a completely different style of work. In retrospect, I truly believe I made the right choice with this project, and I was correct in that when the work became tiresome or long winded, it was my passion for the involved technology that spurred me on to work through it. 
In addition to the subject matter, the style of project work suited me well. There was a nice mix of research and development, meaning I could develop my skills in a number of key areas, something which will no doubt benefit me in the future.    \\

I had little experience both planning and managing a real project before this one. The nature of most work in university has led me to perform work in short, intense bursts, committed to an upcoming deadline. Planning a three month project then was something completely different for me, requiring a much more organised, slow and steady approach. In my opinion, I handled this new challenge well; the use of techniques such as a GANTT chart and setting milestones allowed me to pace myself, producing a steady stream of good quality work. It was however not easy, as my natural tendency is to complete work very close to the deadline; I was very anxious to not let myself be lazy in the first few weeks and leave myself with too much work to do later on. 
From the planning and execution of this project, I have gained faith in many techniques used to organise work, and now see how use of various charts and tools can make a worker much more productive and effective. I have also learned how to pace myself in my work, not doing too much or too little at once, which has been fantastic for my overall discipline, and will certainly help me to avoid leaving work too late in future. \\

Perhaps the greatest challenge for me personally happened part-way through the project. It became clear after my initial research and general project planning that the aims and objectives I had defined were a little too vague and unfocussed. This is of course normal, as the aims and objectives are usually submitted before the student has much knowledge of the domain, but having planned to accomodate some of these objectives, when they did in fact change, it had a great impact on the work to be done. It was at this time that I had to show flexibility and durability, and not lose heart when certain work was no longer possible, or other work needed to be expanded. 
To add to this, several technical limitations were imposed upon the project, such as the inability to deploy components of OpenStack, and so I had to find a way of providing more useful deliverable content without the resources I initially expected, and then adapt my workload to deliver it.\\
I take great pride in the fact that I completely re-thought the objectives of the project, using my research, project justifications, and information about who the stakeholders in the project were as well as the new technical limitations to perform an overhaul of the project's identity. I managed to work out a set of aims and deliverables which would still provide use for anyone interested; for example, I decided to create re-usable software to run experiments, as the current OpenStack deployment could not run them at a large enough scale to produce useful results. In doing this, I discovered that I had the ability to think well under pressure, and still deliver quality despite assorted problems and limitations. \\

The actual implementation of each deliverable taught me a great deal about myself. It firstly taught me that I find it challenging to multi-task at a large scale, and by that I mean simultaneously develop code, design experiments, and perform qualitative metric based analysis. The solution was of course better organisation, allowing me to only concern myself with one task at a time. I found that, as the project came together, I developed skills in each area to the point that multi-tasking became much less of a concern, and easing into each task was not as daunting as it had previously been. Experience of each facet of the work was key to my later ability to produce work quickly and effectively, and this has given me confidence that, even if I initially find something difficult, practice really will make perfect. 

Overall, my Final Year Project has been an unforgettable experience. From learning how to properly review literature, introduce concepts and draw useful conclusions from my research, to implementing useful software tools to potentially aid with an EU project, I have felt like my work was of real value, and completing it has been extremely rewarding. I am thankful to the university, and in particular to my supervisor Karim, for a wonderful few months of work. I hope that everything I have learned in this time will hold me in good stead for the future. \\

Finally, I would like to use the lessons I have learned from reflecting upon my experience to provide advice for future students on their own Final Year Projects. I have listed some of these below. 

\begin{itemize}
\itemsep1em
\item Challenge yourself with your project choice. Do something you're going to enjoy, and have interest in, but avoid something you think will be easy. This is what I did, and it was both fun and rewarding! 

\item Enter a project with an open mind; don't worry if what work is to be done is initially unclear. Aims usually have to be defined very early on, but learning by doing is OK, and these aims can be changed later in the project, as long as they are justified. 

\item Use your research. Often, it seems that the 'Background Research' section is seen as some mandatory prelude to the project, but research can be much more than just giving context to a reader. Research can often shape the aims of the project, or even provide useful methods of evaluation. For example, my research included previous evaluations of OpenStack, and I used these as a model in my planning, as well as a form of evaluation by comparing my project to them. 

\item Don't hesitate to ask your supervisor. I had a problem with OpenStack's networking which I thought might be my own mistake, and it took me a long time to ask about it, but when I did, I had the answer, and a fix, straight away. No one expects you to be an expert, and supervisors are there to help; they can save you a lot of time and effort.  

\item Be prepared to compromise on your workload. You will likely enter the project with very high ambitions, but problems and limitations can and will arise, and you must be ready to deliver less than you had initially hoped. Work is more likely to look incomplete to you, as you knew what more you could've done. To anyone else, it will still look like a good, rounded off piece of work. 

\item If you cannot complete your work, think for the future. Paving the way for future work is almost as good as doing the work yourself. This is something I did in designing re-usable experiments to make up for not being able to run them properly myself. 

\item All in all, enjoy the project experience. It is a great learning opportunity, and what you deliver will undoubtedly be something to be proud of. 

\end{itemize}